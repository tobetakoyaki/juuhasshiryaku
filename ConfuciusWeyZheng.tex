\documentclass[a4j,landscape,twocolumn]{tarticle}
\usepackage{octopus-kanbun}
%\usepackage{layout} % レイアウト表示

% 字の合成

% 入力参考
% \kundoku{字}{よみ}{送り}{訓点}<左送り>(句読点)

\begin{document}
\twocolumn[\begin{center}
	\textbf{\LARGE 十八史略ゼミ 発表資料(文章編)}\\
	孔子・衛・鄭
\end{center}
\begin{flushright}
	文責 ト部蛸焼\\
	第一版 \rensuji{2019}年\rensuji{3}月\rensuji{6}日
\end{flushright}
\vspace{\baselineskip}]

\section{魯・孔子}
\sent
\furi{こう}{孔}\okuri[し]{子}{ハ}(、)\okuri{名}{ハ}\furi{きう}{丘}(、)\okuri{字}{ハ}\furi{ちゅう}{仲}\furi{ぢ}{尼}(。)
\okuri{其}{ノ}\okuri{先}{ハ}\okuri{宋}{ノ}人也。
\ni{有}{リ}\furi{せい}{正}\furi{かう}{考}\okuri[ほ]{父}{トイフ}\ichi{者}{}(、)\re{佐}{タリ}\okuri{宋}{ニ}(。)
\footnotekanbun{三命}{三回官職に命じられる。文献\cite{shoukou7}にとると、正考父は戴公・武公・宣公の三人に仕えた。}三\okuri{命}{シテ}\furi{ます}{滋}\furi{ます}{益}\okuri{恭}{シ}(。)
\okuri{其}{ノ}\chuuskip\footnotekanbun{鼎銘}{宗廟に置く鼎に彫った銘文。}鼎\okuri{銘}{ニ}\okuri{云}{フ}(、)
「一\okuri{命}{シテ}而\okuri[ろう]{僂}{ス}(。)
再\okuri{命}{シテ}而\okuri[う]{傴}{ス}(。)
三\okuri{命}{シテ}而\okuri{俯}{ス}(。)
\re{循}{ヒテ}\okuri[かき]{墻}{ニ}而\okuri{走}{ル}(。)
亦\ni{莫}{シ}\okuri{余}{ヲ}\okuri{敢}{ヘテ}\ichi{侮}{ルコト}(。)
\footnotekanbun{\UTF{9958}・粥}{\UTF{9958}は元々濃い粥を表し、粥は本来薄いものを表す。}\chuuskip\ni[せん]{\UTF{9958}}{シ}於\ichi{是}{ニ}(、)\ni{粥}{シ}於\ichi{是}{ニ}(、)\okuri{以}{テ}\chuuskip\footnotekanbun{餬予口}{糊口を凌ぐ。}\chuuskip \ni[くちすぎ]{餬}{ス}\okuri{予}{ノ}\ichi{口}{ヲ}(。」)
孔氏\ni{滅}{ブ}於\ichi{宋}{ニ}(。)
\okuri{其}{ノ}後\re{適}{ク}\okuri{魯}{ニ}(。)
\ni{有}{リ}\furi{しゅく}{叔}\furi{りやう}{梁}\okuri[こつ]{\UTF{7D07}}{トイフ}\ichi{者}{}(、)\ni{与}{}\footnotekanbun{顔氏女}{徴在を指す。}顔\okuri{氏}{ノ}\ichi[むすめ]{女}{}(、)\ni{\UTF{79B1}}{リテ}於\footnotekanbun{尼山}{現 山東省曲阜市南東にある山。尼丘山とも。}\chuuskip\furi{ぢ}{尼}\ichi[ざん]{山}{ニ}(、)而\ni{生}{ム}孔\ichi{子}{ヲ}(。)
\re{為}{リシトキ}児\footnotekanbun{嬉戯}{遊びたわむれる。「嬉」は遊ぶの意味。}嬉\okuri{戯}{スルニ}(、)\okuri{常}{ニ}\ni[つら]{陳}{ネ}\chuuskip\footnotekanbun{俎豆}{祭りの供物を盛る器具。祭器。}俎\ichi[とう]{豆}{ヲ}(、)\footnotekanbun{設礼容}{礼儀正しい態度をまねする。}\ni{設}{ク}礼\ichi{容}{ヲ}(。)
\okuri{長}{ジテ}\ni{為}{ル}\footnotekanbun{季氏吏}{文献\cite{shiki}によると、米穀の出し入れを司る官職であった委吏のこととされる。}季\okuri{氏}{ノ}\ichi{吏}{ト}(。)
\footnotekanbun{料量}{枡で量る。}料\okuri{量}{ハ}\okuri{平}{ラナリ}(。)
\okuri{嘗}{テ}\ni{為}{ル}\footnotekanbun{司\UTF{6A34}吏}{牧畜を司った役人。}\chuuskip\furi{し}{司}\okuri[しょく]{\UTF{6A34}}{ノ}\ichi[り]{吏}{ト}(。)
畜\footnotekanbun{蕃息}{しげりふえる。}\chuuskip\furi{はん}{蕃}\okuri[そく]{息}{ス}(。)
\footnotekanbun{適周}{南宮敬叔の発案で魯の昭公と共に行った。}\re{適}{キ}\okuri{周}{ニ}\ni{問}{フ}\okuri{礼}{ヲ}於老\ichi{子}{ニ}(。)
\okuri[かへ]{反}{リテ}而弟子\okuri{稍}{ク}益\ninojiten \okuri{進}{ム}(。)
\re{適}{ク}\okuri{斉}{ニ}(。)
\okuri{斉}{ノ}景公\san{将}{ニ}\okuri{待}{スルニ}\ni{以}{テセント}\chuuskip\footnotekanbun{季孟之間}{魯の三桓の季氏(上卿)と孟氏(下卿)との中間。}季・孟之\ichi{間}{ヲ}(。)
孔子\re{反}{ル}\okuri{魯}{ニ}(。)
定公\re{用}{ヰテ}\okuri{之}{ヲ}\re{不}{}\okuri{終}{ヘ}(。)

\yaku{孔子は名を丘、字を仲尼という。その祖先は宋の人であった。(祖先の中に)正考父という人物がおり、宋公に補佐した。彼は三たび官職に命じられ、その度ごとに恭しい態度になっていった。宗廟に置く鼎に彫られた銘文には、「一度目に官職に命じられては僂し、二度目に命じられては傴し、三度目に命じられては俯す。垣根に沿って小走りになって走った。(そのような腰の低い態度でも)決して私を侮蔑するものはなかった。この鼎に濃い粥を盛り、この鼎に薄い粥を盛り、糊口をしのいだ。」とある。孔氏は宋で滅んで、後裔は魯に逃げた。(その後裔に)叔梁\UTF{7D07}という人物がおり、顔氏の娘と尼山で祈祷して孔子が生まれた。孔子がまだ子供のとき、遊ぶときにはいつも祭りの供物を盛る俎と豆を並べて礼儀正しい態度をまねした。成長して(米穀の出し入れを司る官職の)季氏吏になると、米穀の計量は偏りなく正確だった。かつて(牧畜を司る官職の)司\UTF{6A34}吏になると、家畜は大いに繁殖した。周に行き、礼を老子に問うた。魯に帰ると弟子はますます増えた。(その後、魯を離れ)斉に行くと、斉の景公は魯の三桓の季氏と孟氏の間くらいの待遇をするものの、(結局、それを喜ばぬものがおり)孔子は再び魯に帰ってきた。(魯の)定公は最期まで孔子を登用しつづけることはなかった。}

\sent
\re{適}{キ}\okuri{衛}{ニ}(、)\re{将}{ニ}\re{適}{カント}\okuri{陳}{ニ}(。)
\re{過}{グ}\footnotekanbun{匡}{現 河南省新郷市長垣県の西南。}\okuri{匡}{ヲ}(。)
匡人\okuri{嘗}{テ}\ni{為}{ル}\footnotekanbun{陽虎}{魯の人物。初め季平子に仕えたが反乱を起こし、敗れて斉や晋などを巡った。}\chuuskip\furi{やう}{陽}\okuri[こ]{虎}{ノ}\ichire{所}{ト}\okuri{暴}{スル}(。)
孔\okuri{子}{ノ}\furi{かほ}{貌}\ni[に]{類}{テ}陽\ichi{虎}{ニ}(、)\re{止}{ム}\okuri{之}{ヲ}(。)
\okuri{既}{ニシテ}\okuri[のが]{免}{レ}\ni{反}{ル}于\ichi{衛}{ニ}(。)
\ni{醜}{トシ}霊\okuri{公}{ノ}\ichire{所}{ヲ}\okuri{為}{ス}\re{去}{ル}\okuri{之}{ヲ}(。)
\re{過}{ギ}\okuri{曹}{ヲ}\re{適}{キ}\okuri{宋}{ニ}(、)\ni{与}{}弟\ichi{子}{}\ni{習}{フ}\okuri{礼}{ヲ}大\okuri{樹}{ノ}\ichi{下}{ニ}(。)
\furi{くわん}{桓}\furi{たい}{\UTF{9B4B}}\NI[き]{伐}{リ}\okuri{抜}{ク}\okuri{其}{ノ}\ichi{樹}{ヲ}(。)
\re{適}{ク}\okuri{鄭}{ニ}(。)
鄭人\okuri{曰}{ハク}(、)
「東\okuri{門}{ニ}\re{有}{リ}\okuri{人}{}(、)
\okuri{其}{ノ}\okuri[ひたひ]{\UTF{9859}}{ハ}\re{似}{テ}\okuri{堯}{ニ}(、)\okuri{其}{ノ}\okuri[くび]{項}{ハ}\ni[に]{類}{テ}\footnotekanbun{皐陶}{舜の名臣。}\chuuskip\furi{かう}{皐}\ichi[えう]{陶}{ニ}(、)\okuri{其}{ノ}\okuri{肩}{ハ}\ni[に]{類}{タリ}\furi{し}{子}\ichi[さん]{産}{ニ}(。)
\re{自}{リ}\furi{こし}{要}以下、\re{不}{ルコト}\re{及}{バ}\okuri{禹}{ニ}三\okuri{寸}{ニシテ}(、)\chuuskip\footnotekanbun{\UTF{7E8D}\UTF{7E8D}然}{疲れ果てたさま。意気消沈したさま。}\chuuskip\furi{るゐ}{\UTF{7E8D}}\furi{るゐ}{\UTF{7E8D}}\okuri{然}{トシテ}\ni{若}{シト}喪家之\ichi{狗}{ノ}(。」)
\re{適}{キ}\okuri{陳}{ニ}(、)又\re{適}{キ}\okuri{衛}{ニ}(、)\san{将}{ニ}\okuri{西}{ノカタ}\ni{見}{ント}\chuuskip\footnotekanbun{趙簡子}{趙鞅のこと。晋の政治家。}\chuuskip\furi{てう}{趙}\furi{かん}{簡}\ichi[し]{子}{ヲ}(。)
\re{至}{リ}\okuri{河}{ニ}(、)\ni{聞}{キ}\footnotekanbun{竇鳴犢・舜華}{二人共に晋の国の賢大夫。}\chuuskip\furi{とう}{竇}\furi{めい}{鳴}\furi{とく}{犢}・\furi{しゅん}{舜}\okuri[くわ]{華}{ノ}\okuri{殺}{サレテ}\ichi{死}{スヲ}(、)\re{臨}{ミテ}\okuri{河}{ニ}\okuri{歎}{ジテ}\okuri{曰}{ハク}(、)
「\okuri{美}{ナル}哉水、\footnotekanbun{洋洋乎}{水などが満ちていて美しいさま。}洋洋\okuri{乎}{タリ}(。)
\okuri{丘}{ガ}之\re{不}{ルハ}\okuri{済}{ラ}(、)\okuri{此}{レ}命\okuri{也}{ト}(。」)
\ni{反}{リ}于\ichi{衛}{ニ}(、)\re{適}{キ}\okuri{陳}{ニ}(、)\re{適}{キ}\okuri{蔡}{ニ}(、)\re{如}{キ}\footnotekanbun{葉}{楚の国の地名。現 河南省葉県。}\chuuskip\okuri[せふ]{葉}{ニ}(、)\ni{反}{ル}于\ichi{蔡}{ニ}(。)

\yaku{孔子は衛に行き、その後は陳に行こうとして、匡という邑を通過した。匡の人はかつて陽虎に乱暴をされたことがあった。孔子は容貌が陽虎に似ていたため、(匡の人は)孔子を止めた。暫くして(嫌疑は晴れ)逃れて衛に戻った。孔子は(衛の)霊公の行動を醜いとして衛を去った。曹を通過し、宋に行き、そこで弟子と礼を大樹の下で習った。桓\UTF{9B4B}がその木を切って抜いて(孔子を圧死させようとした)。(難を逃れた孔子は)鄭に行った。鄭の人は(孔子の姿を見て)「城下の東門に人がいた。その額は尭に似て、その首元は舜の名臣の皐陶に似て、その肩は子産に似ていた。腰から下は禹に三寸ほど及ばないくらいであり、疲れ果てた様子で家なき犬のようだ。」と言った。孔子は陳に行き、また衛に行き、西の方へ趙簡子に会いに行こうとした。黄河に至り、竇鳴犢や舜華が殺されて死んだことを聞いて、孔子は黄河に臨んで嘆き「美しいかな、この水よ。満ちていて美しい。私が黄河を渡らないのは、これは運命なのだ。」と言った。孔子は衛に帰り、陳に行き、蔡に行き、葉に行ったが蔡に戻ってきた。}

\sent
楚\ni{使}{ム}\okuri{人}{ヲシテ}\ichire{聘}{サ}\okuri{之}{ヲ}(。)
陳・\okuri{蔡}{ノ}大夫\okuri{謀}{リテ}\okuri{曰}{ハク}(、)
「孔子\ni{用}{ヰラルレバ}於\ichi{楚}{ニ}(、)\okuri{則}{チ}陳・\okuri{蔡}{ハ}\okuri{危}{フカラント}矣。」
相\okuri{与}{ニ}\re{発}{シ}\okuri{徒}{ヲ}(、)\ni{囲}{ム}\okuri{之}{ヲ}於\ichi{野}{ニ}(。)
孔子\okuri{曰}{ハク}(、)
「\okuri{詩}{ニ}\okuri{云}{フ}(、)『\footnotekanbun{匪\UTF{5155}匪虎、率彼曠野}{時の為政者に用いられず流浪の旅をしている賢人や長期出征している兵士の嘆き。\UTF{5155}は野牛に似た獣。この一節は『詩経』小雅・何草不黄の篇に見られる。}\re{匪}{ズ}\okuri[じ]{\UTF{5155}}{ニ}\re{匪}{ズ}\okuri{虎}{ニ}(、)\ni[したが]{率}{フト}\okuri[か]{彼}{ノ}\furi{くわう}{曠}\ichi[や]{野}{ニ}(。』)
\okuri{吾}{ガ}\okuri{道}{ハ}\okuri{非}{ナル}\furi{か}{邪}(。)
吾何\okuri{為}{レゾ}\re{於}{イテスルト}\okuri{是}{ニ}(。」)
\furi{し}{子}\furi{こう}{貢}\okuri{曰}{ハク}(、)「夫\okuri{子}{ノ}\okuri{道}{ハ}至\okuri{大}{ナリ}(。)天\okuri{下}{ニ}\ni{莫}{シト}\okuri{能}{ク}\ichi{容}{ルルコト}(。」)
\furi{がん}{顔}\furi{くわい}{回}\okuri{曰}{ハク}(、)「\re{不}{ルコト}\okuri{容}{レ}\okuri{何}{ゾ}\okuri{病}{ヘン}(。)\okuri{然}{ル}後\ni{見}{ルト}君\ichi{子}{ヲ}(。」)
\okuri{楚}{ノ}昭王\re{興}{シテ}\chuuskip\footnotekanbun{師}{二千五百人の軍隊。}\okuri{師}{ヲ}\re{迎}{フ}\okuri{之}{ヲ}(。)
\okuri{乃}{チ}\re{得}{タリ}\re{至}{ルヲ}\re{楚}{ニ}(。)
\san{将}{ニ}\okuri{封}{ズルニ}\ni{以}{テセント}\chuuskip\footnotekanbun{書社之地}{二十五戸分の土地。}書社之地七百\ichi{里}{ヲ}(。)
令尹\furi{し}{子}\furi{せい}{西}\re{不}{}\okuri{可}{カ}(。)
孔子\ni{反}{ル}于\ichi{衛}{ニ}(。)
季康子\okuri{迎}{ヘテ}\re{帰}{ル}\okuri{魯}{ニ}(。)
哀公\re{問}{フ}\okuri{政}{ヲ}(。)
\okuri{終}{ニ}\re{不}{}\re{能}{ハ}\okuri{用}{ヰルコト}(。)

\yaku{楚の君王が使いを出して孔子を招聘させた。陳と蔡の大夫が共謀して「孔子が楚に用いられれば、陳や蔡の国の命運は危ういであろう。」と言った。互いに軍を出し、孔子を野に囲んだ。孔子は「詩経に『\UTF{5155}でもなく虎でもないのに、あの広い荒れ野にさまよう。』とあり(まさに今の私のようである)。さまよっているのは私の説く道が間違っているからなのか。私はどうしてこんなことになるのだろう。」と言った。子貢は「先生の説く道は極めて大きいから、この世に受け入れられることができないのだ。」と言った。顔回は「受け入れられないことをどうして心配することがあろうか。それでこそはじめて君子となるのだ。」と言った。楚の昭王が軍を起こして孔子を迎えに行った。こうして、楚に至ることができた。楚の昭王は孔子を書社の地七百里を与えようとしたが、宰相の子西はそれを聞き入れなかった。孔子は衛に帰った。季康子が迎えて孔子は魯に帰った。魯の哀公が政治について問うたが、結局、孔子を登用することはできなかった。}

\sent
\okuri{乃}{チ}\chuuskip\footnotekanbun{序}{順を追って述べる。}\re{序}{シテ}\okuri{書}{ヲ}(、)\okuri{上}{ハ}\ni{自}{リシテ}唐・\ichi{虞}{}(、)\okuri{下}{ハ}\ni{至}{ル}秦\ichi{繆}{ニ}(。)
\footnotekanbun{刪}{選定する。}\chuuskip\ni[さだ]{刪}{メテ}古詩三\ichi{千}{ヲ}(、)\ni{為}{シ}三百五\ichi{篇}{ト}(、)皆\NI{絃}{}\okuri{歌}{ス}\ichi{之}{ヲ}(。)
礼楽\re{自}{リ}此\re{可}{シ}\okuri{述}{ブ}(。)
\okuri{晩}{ニシテ}而\re[この]{喜}{ミ}\okuri{易}{ヲ}(、)\ni{序}{ス}\footnotekanbun{彖・象・\UTF{7E6B}辞・説卦・文言}{全て書物の名。『彖伝』は『周易』の卦の意味を総論する。『象伝』は各卦と爻の意味を示す。『\UTF{7E6B}辞伝』は易の成立や思想などを伝える。『説卦伝』は八卦を語る。『文言伝』は六十四卦のうちの重要な二つである乾と坤の解釈を伝える。}\chuuskip\furi{たん}{彖}・\furi{しやう}{象}・\furi{けい}{\UTF{7E6B}}\furi{じ}{辞}・\furi{せっ}{説}\furi{くわ}{卦}・\furi{ぶん}{文}\ichi[げん]{言}{ヲ}(。)
\re{読}{ミ}\okuri{易}{ヲ}(、)韋編\okuri{三}{タビ}\okuri{絶}{ツ}(。)
\ni{因}{リテ}\okuri{魯}{ノ}史\ichi{記}{ニ}(、)\ni{作}{ル}春\ichi{秋}{ヲ}(。)
\re{自}{リ}隠\re{至}{ルマデ}\okuri{哀}{ニ}十二\okuri{公}{ニシテ}(、)\ni{絶}{ツ}\okuri{筆}{ヲ}於獲\ichi{麟}{ニ}(。)
\okuri{筆}{スベキハ}\okuri{則}{チ}\okuri{筆}{シ}(、)\okuri{削}{ルベキハ}\okuri{則}{チ}\okuri{削}{ル}(。)
\footnotekanbun{子夏}{文学に通じた孔子の弟子。}\chuuskip\furi{し}{子}\furi{か}{夏}之徒\re{不}{}\ni{能}{ハ}\okuri[たす]{賛}{クルコト}一\ichi{辞}{モ}(。)
弟\okuri{子}{ハ}三千人。身\ni{通}{ズル}\chuuskip\footnotekanbun{六芸}{礼・楽・射・御・書・数。}\chuuskip\furi{りく}{六}\ichi[げい]{芸}{ニ}者、七十有二\okuri{人}{ナリ}(。)
\furi{よはひ}{年}七十\okuri{三}{ニシテ}而\okuri{卒}{ス}(。)
\okuri{子}{ノ}\furi{り}{鯉}、\okuri{字}{ハ}\furi{はく}{伯}\furi{ぎょ}{魚}(、)\okuri{早}{クニ}\okuri{死}{ス}(。)
\okuri{孫}{ノ}\furi{きふ}{\UTF{4F0B}}(、)\okuri{字}{ハ}\furi{し}{子}\furi{し}{思}(、)\ni{作}{ル}中\ichi{庸}{ヲ}(。)

\yaku{そこで孔子は書経の順序を正しくし、上は尭・舜から下は秦の穆王まで至った。三千の古詩の中から選定して三百五篇として、すべてを絃楽器に合わせて歌えるようにした。これから(先王の)礼楽が述べられるようになった。孔子は晩年に易を好み、彖・象・\UTF{7E6B}辞・説卦・文言を著した。易を(熱心に)読むあまり、その鞣し革の綴じ糸は三回も切れた。孔子はさらに魯の歴史書に基づいて、春秋を作った。(魯の)隠公から(魯の)哀公に至るまでの十二人の君王で「獲麟」の二字で筆を置いた。書き加えるべきところは書き加え、削り取るべきところは削り取り、(文学に通じた)子夏のような弟子でも一字も付け加えることはできなかった。(それほどに完璧であった。)孔子は弟子が三千人も居たが、そのうち六芸に通じている者は七十二人であった。孔子は七十三歳で亡くなった。子の鯉、字は伯魚は早くに亡くなった。孫の\UTF{4F0B}、字は子思は、中庸を作った。}

\sent
\furi{まう}{孟}\okuri[し]{子}{ハ}\okuri{其}{ノ}門人也。
\okuri{名}{ハ}\furi{か}{軻}(、)\okuri{魯}{ノ}孟孫之\okuri{後}{ナリ}(。)
\ni{生}{マル}於\ichi[すう]{鄒}{ニ}(。)
\okuri{幼}{ニシテ}\ni{被}{リ}慈\okuri{母}{ニ}三遷之\ichi{教}{ヲ}(、)\okuri{長}{ジテ}\ni{受}{ク}\okuri{業}{ヲ}子思之\ichi{門}{ニ}(。)
道\okuri{既}{ニシテ}\okuri{通}{ジ}(、)\ni{游}{ブ}斉・\ichi{梁}{ニ}(。)\re{不}{}\okuri{用}{ヰラレ}(。)
\re{退}{キテ}\ni{与}{}\furi{ばん}{万}\furi{しやう}{章}之\ichi{徒}{}(、)\footnotekanbun{難疑}{欠点を非難し、疑問点を研究する。}難疑答\okuri{問}{シテ}\ni{作}{ル}七\ichi{篇}{ヲ}(。)

\yaku{孟子は子思の門下生で、名は軻、魯の孟孫氏の後裔であり、鄒に生まれた。幼い頃、母親から(子のためを思って養育環境を三回変えるという)三遷の教えを受け、成長して子思の門に学業を学んだ。すぐに道に通じ、斉や梁の地へ赴いたが、政治に登用されなかった。孟子は弟子の万章の一行とともに故郷へ戻り、討論や問答をして『孟子』七篇を作った。}

\sent
\furi{らう}{老}\furi{し}{子}\furi{は}{者}(、)\okuri{楚}{ノ}\furi{こ}{苦}\okuri[けん]{県}{ノ}人也。
李姓、\okuri{名}{ハ}\furi{じ}{耳}(、)\okuri{字}{ハ}\furi{はく}{伯}\furi{やう}{陽}(。)又\ni{曰}{フ}\okuri{字}{ヲ}\ichi[たん]{\UTF{803C}}{トモ}(。)
\ni{為}{ル}\okuri{周}{ノ}\chuuskip\footnotekanbun{守蔵吏}{金蔵を管理する役人。}守蔵\ichi{吏}{ト}(。)
孔子\re{問}{フ}\okuri{焉}{ニ}(。)
老子\re{告}{ゲテ}\okuri{之}{ニ}\okuri{曰}{ハク}(、)
「\furi{りやう}{良}\okuri[こ]{賈}{ハ}\okuri{深}{ク}\okuri{蔵}{シテ}\re{若}{ク}\okuri{虚}{シキガ}(、)君\okuri{子}{ハ}盛\okuri{徳}{アルモ}容貌\re{若}{シト}\okuri{愚}{ナル}(。」)
孔子\okuri{去}{リテ}\ni{謂}{ヒテ}弟\ichi{子}{ニ}\okuri{曰}{ハク}(、)
「\okuri{鳥}{ハ}吾\ni{知}{ル}\okuri{其}{ノ}\okuri{能}{ク}\ichi{飛}{ブヲ}(。)
\okuri{魚}{ハ}吾\ni{知}{ル}\okuri{其}{ノ}\okuri{能}{ク}\ichi{游}{グヲ}(。)
\okuri{獣}{ハ}吾\ni{知}{ル}\okuri{其}{ノ}\okuri{能}{ク}\ichi{走}{ルヲ}(。)
\okuri{走}{ル}\okuri{者}{ハ}\ni{可}{ク}\okuri{以}{テ}\ichire{為}{ス}\okuri{網}{ヲ}(、)\okuri{游}{グ}\okuri{者}{ハ}\ni{可}{ク}\okuri{以}{テ}\ichire{為}{ス}\footnotekanbun{綸}{釣り糸。}\chuuskip\okuri[いと]{綸}{ヲ}(、)\okuri{飛}{ブ}\okuri{者}{ハ}\ni{可}{シ}\okuri{以}{テ}\ichire{為}{ス}\okuri[いぐるみ]{\UTF{77F0}}{ヲ}(。)
\ni{至}{リテハ}於\ichi{竜}{ニ}(、)吾\re{不}{}\re{能}{ハ}\okuri{知}{ルコト}(。)
\okuri{其}{レ}\ni{乗}{ジテ}風\ichi{雲}{ニ}而\re{上}{ル}\okuri{天}{ニ}也。
今\ni{見}{ルニ}老\ichi{子}{ヲ}(、)\okuri{其}{レ}\re{猶}{ホ}<キ>\okuri{竜}{ノ}\okuri[かな]{乎}{ト}(。」)
老子\ni{見}{テ}\okuri{周}{ノ}\ichi{衰}{ヘタルヲ}(、)\okuri{去}{リテ}\re{至}{ル}\okuri{関}{ニ}(。)
関令\furi{ゐん}{尹}\furi{き}{喜}\okuri{曰}{ハク}(、)
「\furi{し}{子}\re{将}{ニ}\okuri{隠}{レント}矣、\re{為}{ニ}\okuri{我}{ノ}\re{著}{セト}\okuri{書}{ヲ}(。」)
\okuri{乃}{チ}\ni{著}{シテ}道徳五千余\ichi{言}{ヲ}而\okuri{去}{ル}(。)
\re{莫}{シ}\ni{知}{ル}\okuri{其}{ノ}\ichire{所}{ヲ}\okuri{終}{フル}(。)
\okuri{其}{ノ}後\ni{有}{リ}鄭人\furi{れつ}{列}\furi{ぎょ}{禦}\furi{こう}{寇}・\footnotekanbun{蒙}{宋の邑。}蒙人\furi{さう}{荘}\ichi[しう]{周}{}(。)
亦\ni{為}{ス}老子之\ichi{学}{ヲ}(。)
荘周\re{著}{シ}\okuri{書}{ヲ}(、)\ni{侮}{リテ}孔\ichi{子}{ヲ}(、)而\ni[せ]{誚}{ム}諸\ichi{子}{ヲ}焉。

\yaku{老子は楚の苦県に人であり、李姓で名は耳、字は伯陽、字は\UTF{803C}ともいう。周の金蔵を管理する官職である守蔵吏になる。孔子は彼に(礼を)問うた。老子が孔子に「大商人は商品を人目のつかない奥にしまい、一見何もないように見せかける。君子は立派な徳を有していても、外見を気にかけないので、一見、愚者のように見える。」と告げた。孔子は(老子の許を)去って弟子に「鳥はよく空を飛ぶものだと私は知っている。魚はよく泳ぐものだと私は知っている。獣はよく大地を走るものだと私は知っている。大地を走るものは網で捕まえることができ、泳ぐものは釣り糸で捕まえることができ、飛ぶものはいぐるみで捕まえることができる。しかし、竜に至っては、私はよく分からない。竜は風雲に乗って天に上ってしまうからである。いま、老子を見ると、彼はまるで竜のような(とらえどころのない)存在だなあ。」と言った。老子は周室が衰えてゆくのを見て、周を去り函谷関に至った。そこの番人の尹喜は「あなたは隠遁しようとしているけれども、どうか私のために本を書いてください。」と言った。そこで老子は五千あまりの道徳の言葉を書き記して函谷関を去った。その後の老子の行方は知られていない。その後、鄭に列禦寇、蒙に荘周という人物がそれぞれいた。彼らもまた老子の学問を修めた。荘周は本を書いて孔子を侮蔑し、孔子の弟子らを責めた。}

\section{衛}
\sent
\okuri{衛}{ハ}姫姓、武\okuri{王}{ノ}\chuuskip\footnotekanbun{母弟}{同じ母から生まれた弟。}母弟、\furi{かう}{康}\furi{しゅく}{叔}\furi{ほう}{封}之\re{所}{}\okuri{封}{ゼラレシ}也。
後世\ni{至}{リ}春\ichi{秋}{ニ}(、)\ni{有}{リ}霊\okuri{公}{ノ}夫人\furi{なん}{南}\furi{し}{子}之\ichi{乱}{}(。)
\okuri{子}{ノ}\furi{くわい}{\UTF{84AF}}\furi{くわい}{\UTF{8075}}\re{欲}{スルモ}\ni{殺}{サント}南\ichi{子}{ヲ}(、)\re{不}{シテ}\okuri{果}{タサ}出\okuri{奔}{ス}(。)
公\okuri{卒}{ス}(。)
\ni{立}{ツ}\UTF{84AF}\UTF{8075}之\okuri{子}{ノ}\ichi[てふ]{輒}{}(。)
\UTF{84AF}\UTF{8075}\okuri{入}{ル}(。)
輒\re[ふせ]{拒}{グ}\okuri{之}{ヲ}(。)
\furi{し}{子}\furi{ろ}{路}\ni{与}{ル}\okuri{其}{ノ}\ichi{難}{ニ}(。)
\footnotekanbun{太子之臣}{\UTF{84AF}\UTF{8075}の臣の石乞(せきこつ)と于黶(うえん)。}太子之臣、\re{以}{テ}\okuri{戈}{ヲ}\ni{撃}{チ}子\ichi{路}{ヲ}(、)\re{断}{ツ}\okuri{纓}{ヲ}(。)
子路\okuri{曰}{ハク}(、)「君\okuri{子}{ハ}\okuri{死}{ストモ}\okuri{冠}{ハ}\re{不}{ト}\okuri[ぬ]{免}{ガ}(。」)
\re{結}{ビテ}\okuri{纓}{ヲ}而\okuri{死}{ス}(。)
衛人\ni{醢}{ニス}子\ichi{路}{ヲ}(。)
孔子\re{聞}{キテ}\okuri{之}{ヲ}(、)\okuri{命}{ジテ}\re{覆}{サシム}\okuri{醢}{ヲ}(。)

\yaku{衛は(周と同じ)姫姓であり、武王の同母弟の康叔封の封ぜられた場所である。後の世、春秋に至って、霊公の夫人南子の騒動が起こった。霊公の子の\UTF{84AF}\UTF{8075}が南子を殺そうとしたが、失敗し出奔した。霊公が亡くなり、\UTF{84AF}\UTF{8075}の子の輒が君王になった。\UTF{84AF}\UTF{8075}は衛に入ろうとしたが、輒がこれを防いだ。子路はその難に関係した。\UTF{84AF}\UTF{8075}の家臣らが戈で子路を斬りつけ冠の紐を断った。子路は「君子は死んでも冠は脱がないものだ。」と言って、冠の紐を結んでいる間に死んだ。衛の人は子路を塩からにして(罰した)。孔子はこのことを聞いて(同情し)、(自分の家の)塩からの入れ物をひっくり返して塩からを捨てさせた。}

\sent
戦\okuri{国}{ノ}時、\furi{し}{子}\furi{し}{思}\ni{居}{ル}於\ichi{衛}{ニ}(。)
\ni{言}{フ}\furi{かう}{苟}\okuri[へん]{変}{ヲ}\ichire{可}{シト}\chuuskip\footnotekanbun{将}{軍隊を率いる人。}\okuri{将}{ニス}(。)
衛侯\okuri{曰}{ハク}(、)
「\okuri{変}{ハ}\okuri{嘗}{テ}\re{為}{リシトキ}吏、\footnotekanbun{賦}{取り立てる。}\ni{賦}{シテ}於\ichi{民}{ニ}(、)\ni{食}{フ}\okuri{人}{ノ}二\UTF{96DE}\ichi{子}{ヲ}(。)\okuri{故}{ニ}\re{弗}{ト}\okuri{用}{ヰ}(。」)
子思\okuri{曰}{ハク}(、)
「聖\okuri{人}{ノ}\re{用}{ヰルコト}\okuri{人}{ヲ}(、)\ni{猶}{ホ}<シ>匠之\ichire{用}{ヰルガ}\okuri{木}{ヲ}(。)
\ni{取}{リ}\okuri{其}{ノ}\ichire{所}{ヲ}\okuri{長}{キ}(、)\ni{棄}{ツ}\okuri{其}{ノ}\ichire{所}{ヲ}\okuri{短}{キ}(。)
\okuri{故}{ニ}\chuuskip\footnotekanbun{杞梓}{おうちとあずさの木。共に器物の良材。}杞\okuri{梓}{ハ}\chuuskip\footnotekanbun{連抱}{両手で抱える。}連\okuri{抱}{シテ}而\ni{有}{ルトモ}数尺之\ichi{朽}{}(、)良工\re{不}{}\okuri{棄}{テ}(。)
今君\ni{処}{リテ}戦国之\ichi{世}{ニ}(、)而\ni{以}{テ}二\ichi{卵}{ヲ}\ni{棄}{ツ}\footnotekanbun{干城之将}{君主の盾となり城となって外を防ぎ内を守る将軍。}干城之\ichi{将}{ヲ}(。)
\okuri{此}{レ}\re{不}{ル}\re{可}{カラ}\re{使}{ムル}\ni{聞}{カセ}於隣\ichi{国}{ニ}\okuri{也}{ト}(。」)

\yaku{戦国のとき、子思は衛に仕えていた。苟変を大将にするべきだと衛の君王に言った。衛の慎公は「苟変はかつて官吏であったときに民から卵を二つずつ取り立てて食べたことがある。だから、彼は登用しない。」と返した。すると子思は、「聖人が人を登用することは、ちょうど大工が木を用いるようなものです。つまり、その長じた部分を取って悪い部分は捨てるということです。だから杞や梓のような良材は両手で抱える程度の大きさだと、数尺の腐敗があっても良工はそれを捨てないのです。いま、あなたは戦国の世にあって、(たった)二つの卵が原因で君主の盾となり城となるほどの名将を捨ててしまうのです。これは隣国に聞かせてはならない(ほど恥ずかしいことなのです)。」と諫言した。}

\sent
衛侯\re{言}{フ}\okuri{計}{ヲ}(。)
\re{非}{ズ}\okuri{是}{ニ}(。)
\okuri{而}{シテ}群\okuri{臣}{ノ}\okuri{和}{スル}者、\re{如}{シ}\ni{出}{ヅルガ}一\ichi{口}{ニ}(。)
子思\okuri{曰}{ハク}(、)
「君之国事、\ni{将}{ニ}\okuri{日}{ニ}\ichi{非}{ナラント}矣。
君\re{出}{シテ}\okuri{言}{ヲ}\okuri{自}{ラ}\okuri{以}{テ}\re{為}{シ}\okuri{是}{ト}(、)\okuri{而}{シテ}卿大\okuri{夫}{ハ}\san{莫}{シ}\okuri{敢}{ヘテ}\ni{矯}{ムル}\okuri{其}{ノ}\ichi{非}{ヲ}(。)
卿大夫\re{出}{シテ}\okuri{言}{ヲ}\okuri{自}{ラ}\okuri{以}{テ}\re{為}{シ}\okuri{是}{ト}(、)\okuri{而}{シテ}士庶\okuri{人}{ハ}\san{莫}{シ}\okuri{敢}{ヘテ}\ni{矯}{ムル}\okuri{其}{ノ}\ichi{非}{ヲ}(。)
\okuri{詩}{ニ}\okuri{曰}{フ}(、)『\okuri[とも]{具}{ニ}\ni{曰}{フ}\okuri{予}{ヲ}\ichi{聖}{ナリト}(。)\okuri{誰}{カ}\ni{知}{ラント}烏之雌\ichi{雄}{ヲ}(。』」)
周之諸侯、\okuri{惟}{ダ}\okuri{衛}{ノミ}\okuri{最}{モ}\okuri{後}{ニ}\okuri{亡}{ブ}(。)
\shita{至}{リテ}\okuri{秦}{ノ}\ni{\UTF{5E77}}{セテ}天\ichi{下}{ヲ}\naka{為}{ルニ}\okuri{帝}{ハ}\furi{に}{二}\ue[せい]{世}{ト}(、)\okuri{始}{メテ}\ni{廃}{シテ}\furi{くん}{君}\ichi[かく]{角}{ヲ}\ni{為}{ス}庶\ichi{人}{ト}(。)

\yaku{衛の君王が計略を述べたが、その計略はよしとできないものであったのに、家臣らが調子を合わせることは、まるで一つの口から言葉を発しているようであった。そこで子思は「これでは君の政治は日に日に駄目になっていくばかりです。君王が発言して自らでそれを正しいと決めつけると、卿や大夫は決してその間違いを改めることはありません。卿や大夫が発言して自らでそれを正しいと決めつけると、士や一般の人は決してその間違いを改めることはありません。詩経に「みな口を揃えて自分を聖人だと言っているが、誰も烏の雌雄さえ判別できないではないか。(ましてや自分の素晴らしさや愚かさを判別できるものか。)」とあります。」と言った。周の諸侯のうち、ただ衛だけが一番最後に滅んだ。秦が天下を統一してから帝は二世のときになって初めて、君角を廃して平民とした。}

\section{鄭}
\okuri{鄭}{ハ}姫姓、\okuri{周}{ノ}宣\okuri{王}{ノ}庶弟、\furi{くわん}{桓}\furi{こう}{公}\furi{いう}{友}之\re{所}{}\okuri{封}{ゼラレシ}也。
桓\okuri{公}{ノ}\okuri{子}{ノ}\furi{ぶ}{武}\okuri[こう]{公}{ト}(、)\ni{与}{}\okuri{其}{ノ}\okuri{子}{ノ}\furi{さう}{荘}\ichi[こう]{公}{}(、)\okuri{並}{ビニ}\ni{為}{ル}\okuri{周}{ノ}\chuuskip\footnotekanbun{司徒}{教育を司る官名。}司\ichi{徒}{ト}(。)
数\okuri{世}{シテ}\ni{至}{リ}\furi{せい}{声}\ichi[こう]{公}{ニ}(、)\ni{相}{トス}\furi{し}{子}\ichi[さん]{産}{ヲ}(。)
子産\furi{は}{者}公族、国氏、\okuri{名}{ハ}\furi{けう}{僑}(。)
孔子\re{過}{ギ}\okuri{鄭}{ヲ}(、)\ni{与}{}子\ichi{産}{}\ni{如}{シト}兄\ichi{弟}{ノ}\okuri{云}{フ}(。)
\furi{ぼく}{穆}・\okuri[じやう]{襄}{ヨリ}以来、鄭\san{無}{シ}\okuri{歳}{トシテ}\re{不}{ルコト}\ni{被}{ラ}晋・楚之\ichi{兵}{ヲ}(。)
子産\re{受}{ケ}\okuri{之}{ヲ}(、)\re{以}{テ}\okuri{礼}{ヲ}\okuri{自}{ラ}\okuri{固}{ス}(。)
\ni{雖}{モ}晋・楚之\ichi{暴}{ト}(、)\re{不}{}\re{能}{ハ}\okuri{加}{フル}焉。
鄭\ni{至}{リテ}\okuri{周}{ノ}威烈\okuri{王}{ノ}\ichi{時}{ニ}(、)\furi{くん}{君}\furi{いつ}{乙}\ni{為}{ル}\okuri{韓}{ノ}\furi{あい}{哀}\okuri[こう]{侯}{ノ}\ichire{所}{ト}\okuri{滅}{ボス}(。)
韓\ni{徙}{ス}\okuri{都}{ヲ}\ichi{之}{ニ}(。)

\yaku{鄭は(周と同じ)姫姓であり、周の宣王の腹違いの弟の桓公友の封ぜられたところである。桓公の子の武公とその子の荘公はともに周の教育を司る官職であった司徒であった。そこから数世隔てて声公に至り、彼は子産を宰相とした。子産は鄭の公侯と同族であり、氏は国、名は僑である。孔子が鄭を通過したとき、子産とはまるで兄弟のようだと称した。穆公や襄公からのち、鄭が晋や楚の兵に攻められない年は一年もなかった。子産が国政を受け、礼をもって自国の地位を固めていくと、いくら強暴な晋や楚といえども威力を与えることはできなかった。鄭は周の威烈王のときに至って、君乙が韓の哀侯によって滅ぼされた。韓は都を\UTF{6ECE}陽に遷した。}

\newpage
\appendix
\section{史記に読む詳細}
\sent
孔子\san{知}{リ}弟\okuri{子}{ガ}\ni{有}{ルヲ}\chuuskip\footnotekanbun{慍心}{憤りの心。}\chuuskip\furi{をん}{慍}\ichi[しん]{心}{}(、)
\okuri{乃}{チ}\ni{召}{シテ}子\ichi{路}{ヲ}而\okuri{問}{ヒテ}\okuri{曰}{ハク}(、)
「\okuri{詩}{ニ}\okuri{云}{フ}(、)『\re{匪}{ズ}\okuri[じ]{\UTF{5155}}{ニ}\re{匪}{ズ}\okuri{虎}{ニ}(、)\ni[したが]{率}{フト}\okuri[か]{彼}{ノ}\furi{くわう}{曠}\ichi[や]{野}{ニ}(。』)
\okuri{吾}{ガ}\okuri{道}{ハ}\okuri{非}{ナル}\furi{か}{耶}(。)
吾何\okuri{為}{レゾ}\re{於}{イテスルト}\okuri{此}{ニ}(。)」
子路\okuri{曰}{ハク}(、)
「\furi{お}{意}\okuri[も]{者}{フニ}吾\footnotekanbun{未仁}{真の仁者でない。}\re{未}{ダ}<ル>\okuri{仁}{ナラ}\furi{か}{耶}(。)
人之\ni{不}{ルハ}\okuri{我}{ヲ}\ichi{信}{ゼ}也。
\furi{お}{意}\okuri[も]{者}{フニ}吾\footnotekanbun{未知}{真の知者でない。}\re{未}{ダ}<ル>\okuri{知}{ナラ}\furi{か}{耶}(。)
人之\ni{不}{ルハト}\okuri{我}{ヲ}\ichi{行}{カセ}也。」
孔子\okuri{曰}{ハク}(、)
「\re{有}{ラン}是\furi{や}{乎}(。)
\furi{いう}{由}(、)\okuri{譬}{ヘバ}\ni{使}{メバ}仁\okuri{者}{ヲシテ}而\okuri{必}{ズ}\ichi{信}{ゼラレ}(、)\okuri{安}{ンゾ}\ni{有}{ラン}\chuuskip\footnotekanbun{伯夷・叔斉}{殷末の賢人。周の武王が殷の紂王を討とうとするのを諌め、のち首陽山に隠れ住んで餓死した。}\chuuskip\furi{はく}{伯}\furi{い}{夷}・\furi{しゅく}{叔}\ichi[せい]{斉}{}(。)
\ni{使}{メバ}智\okuri{者}{ヲシテ}而\okuri{必}{ズ}\ichi{行}{ハレ}(、)
\okuri{安}{ンゾ}\ni{有}{ラント}\chuuskip\footnotekanbun{王子比干}{殷の忠臣。紂王のおじで、紂を諌めて怒りにあい、胸を裂かれた。}\chuuskip\furi{わう}{王}\furi{し}{子}\furi{ひ}{比}\ichi[かん]{干}{}(。)」
子路\okuri{出}{ヅ}(。)

\yaku{孔子は弟子たちに憤りの心が有るのを知ったので、子路を招いて問うた。「詩に、『野牛でもなく、虎でもないのに、どうしてこの広野にひき廻さるる!』と、歌っているが、わが道が悪いのであろうか。われはどうしてここに苦しまなければならんのか」と。子路が言った。「思いますに、わたしたちはまだ仁者ではないのでしょう、人がわれわれを信じませんのは!。思いますに、われわれは、まだ智者でないのでしょう、人がわれわれを行かせないのは!」と。孔子が言った。「どうして、そんなことがあるものか。由よ、たとえば、仁者が必ず人に信ぜられるものなら、どうして伯夷・叔斉のような仁人が餓死することが有ろうか。智者が必ず行きたい所へ行きうるなら、どうして王子比干が腹を剖かれるようなことがあろうか」と。子路は退出した。}

\sent
\furi{し}{子}\furi{こう}{貢}\okuri{入}{リテ}\okuri[まみ]{見}{ユ}(。)
孔子\okuri{曰}{ハク}(、)
「賜、\okuri{詩}{ニ}\okuri{云}{フ}(、)『\re{匪}{ズ}\okuri[じ]{\UTF{5155}}{ニ}\re{匪}{ズ}\okuri{虎}{ニ}(、)\ni[したが]{率}{フト}\okuri[か]{彼}{ノ}\furi{くわう}{曠}\ichi[や]{野}{ニ}(。』)
\okuri{吾}{ガ}\okuri{道}{ハ}\okuri{非}{ナル}\furi{か}{耶}(。)
吾何\okuri{為}{レゾ}\re{於}{イテスルト}\okuri{此}{ニ}(。)」
子貢\okuri{曰}{ハク}(、)
「夫子之\okuri{道}{ハ}至大也。
\okuri{故}{ニ}天下\san{莫}{シ}\okuri{能}{ク}\ni{容}{ルル}夫\ichi{子}{ヲ}(。)
夫子\ni{蓋}{ゾ}<ルト>\okuri{少}{シク}\chuuskip\footnotekanbun{貶}{おとす。低くする。}\chuuskip\ichire[へん]{貶}{セ}\okuri{焉}{ヲ}(。」)
孔子\okuri{曰}{ハク}(、)
「賜、良\okuri{農}{ハ}\okuri{能}{ク}\chuuskip\footnotekanbun{稼}{種を播きつける。}\okuri{稼}{ス}(、)\okuri{而}{レドモ}\re{不}{}\re{能}{ハ}\re{為}{スコト}\chuuskip\footnotekanbun{穡}{取り入れる。収穫する。}\re{穡}{ヲ}(。)
良\okuri{工}{ハ}\okuri{能}{ク}\okuri{巧}{ミナリ}(、)\okuri{而}{レドモ}\re{不}{}\re{能}{ハ}\re{為}{スコト}\chuuskip\footnotekanbun{順}{人の意にかなう。}\re{順}{ヲ}(。)
君\okuri{子}{ハ}\okuri{能}{ク}\ni{修}{メ}\okuri{其}{ノ}\ichi{道}{ヲ}(、)\footnotekanbun{綱}{大綱。大きい道。}\okuri{綱}{シテ}而\footnotekanbun{紀之}{それをきまりとし、道筋とする。}\re{紀}{シ}\okuri{之}{ヲ}(、)\footnotekanbun{統而理}{すべおさめる。実行できるようにする。}\okuri{統}{シテ}而\re{理}{ス}\okuri{之}{ヲ}(、)\okuri{而}{レドモ}\re{不}{}\re{能}{ハ}\re{為}{スコト}\chuuskip\footnotekanbun{容}{容認。}\okuri{容}{レラルルヲ}(。)
今爾\re{不}{シテ}\ni{修}{メ}\okuri{爾}{ノ}\ichi{道}{ヲ}\okuri{而}{モ}\re{求}{ム}\re{為}{スヲ}\okuri{容}{レラルルヲ}(。)
賜、\footnotekanbun{而志不遠}{汝の志すところは遠大でない。}\okuri{而}{ノ}\okuri{志}{ハ}\re{不}{ト}\okuri{遠}{カラ}矣。」
子貢\okuri{出}{ヅ}(。)

\yaku{子貢が入ってきて孔子にお目にかかった。孔子が言った。「賜よ、詩に歌っている。『野牛でもなく、虎でもないのに、どうしてこの広野にさまようのか!』と。わが説く道が悪いのか、われはどうしてここにさまよわなければならんのか!」と。子貢が言った。「先生の道は至大であります。だから、天下は先生を容れることができないのです。どうして少しくその道を小さく低くなさいませんか」と。孔子が言った。「賜よ、良農はうまく種を播くが、よく収穫できるとは限らない。良工は器物を作ることは巧みでも、よく人の好みに順うとは限らない。君子はよくその道を修め、大綱をたてて、それを道筋とし、これを統理することはできるが、必ずしも世人に容れられるとは限らない。今、おまえは、おまえの道を修めないで、世人に容れられんことを求めている。賜よ、おまえの志は遠大でないよ」と。子貢は退出した。}

\sent
\furi{がん}{顔}\furi{くわい}{回}\okuri{入}{リテ}\okuri{見}{ユ}(。)
孔子\okuri{曰}{ハク}(、)
「回、\okuri{詩}{ニ}\okuri{云}{フ}(、)『\re{匪}{ズ}\okuri[じ]{\UTF{5155}}{ニ}\re{匪}{ズ}\okuri{虎}{ニ}(、)\ni[したが]{率}{フト}\okuri[か]{彼}{ノ}\furi{くわう}{曠}\ichi[や]{野}{ニ}(。』)
\okuri{吾}{ガ}\okuri{道}{ハ}\okuri{非}{ナル}\furi{か}{耶}(。)
吾何\okuri{為}{レゾ}\re{於}{イテスルト}\okuri{此}{ニ}(。)」
顔回\okuri{曰}{ハク}(、)
「夫子之\okuri{道}{ハ}至\okuri{大}{ナリ}(、)\okuri{故}{ニ}天下\ni{莫}{シ}\okuri{能}{ク}\ichi{容}{ルル}(。)
\re{雖}{モ}\okuri{然}{リト}(、)夫子\okuri{推}{シテ}而\re{行}{ヘ}\okuri{之}{ヲ}(。)
\re{不}{ルハ}\okuri{容}{レラレ}\okuri{何}{ゾ}\okuri{病}{ヘン}(。)
\re{不}{シテ}\okuri{容}{レラレ}\okuri{然}{ル}\okuri{後}{ニ}\ni{見}{ル}君\ichi{子}{ヲ}(。)
\okuri{夫}{レ}\chuuskip\footnotekanbun{夫道之不修也、是吾醜也}{そもそも人たるの道を修めないことが、わたしの恥である。}道之\re{不}{ルハ}\okuri{修}{マラ}也、\okuri{是}{レ}\okuri{吾}{ガ}\furi{はぢ}{醜}也。
\okuri{夫}{レ}道\furi{す}{既}\okuri[で]{已}{ニ}\okuri{大}{イニ}\okuri{修}{マリテ}\okuri{而}{モ}\re{不}{ルハ}\okuri{用}{ヒラレ}(、)\okuri{是}{レ}\re[たも]{有}{ツ}\okuri{国}{ヲ}者之\furi{はぢ}{醜}也。
\re{不}{ルハ}\okuri{容}{レラレ}\okuri{何}{ゾ}\okuri{病}{ヘン}(。)\re{不}{シテ}\okuri{容}{レラレ}\okuri{然}{ル}\okuri{後}{ニ}\ni{見}{ルト}君\ichi{子}{ヲ}(。)」
孔子欣\okuri{然}{トシテ}而\okuri{笑}{ヒテ}\okuri{曰}{ハク}(、)
「\footnotekanbun{有是哉}{そうあるべきだ。期待していたことを聞いて感激した受け言葉。}\re{有}{ル}\okuri{是}{レ}\furi{かな}{哉}(、)顔氏之\okuri{子}{ヨ}(。)
\ni{使}{メバ}\okuri{爾}{ヲシテ}\ichire{多}{カラ}財、吾\ni{為}{ラント}\okuri{爾}{ノ}\chuuskip\footnotekanbun{宰}{家老。今でいえば取締役。}\ichi{宰}{ト}(。)」

\yaku{顔回が入ってきて孔子にお目にかかった。孔子が言った。「回よ、詩に歌っている、『\UTF{5155}でもなく、虎でもない。だのにどうしてこの広野にさまようのか!』と。わしが説く道は間違っているのか。どうして、ここにこの困厄にかかるとは!」と。顔回が言った。「先生の道は至大でございます。ですから、天下によく容るるものがないのでございます。でありますが、先生には是非ともその道を推して行っていただきたいのであります。世人に容れられないことなどは、どうして憂える必要がありましょうや。容れられないでこそ、初めて君子たることがわかるのでございます。道の修まらないことこそ、これはわれわれの恥でございます。道がすでに大いに修まっていて、用いないのは、国を有する君主の恥であります。世に容れられないことは、どうして憂うべきことでございましょうや。むしろ、世に容れられなくてこそ、しかる後に初めて君子たることがわかるのでございます」と。孔子は欣然として笑って言った。「そうあるべきだ、顔氏の子よ。もし、お前が財産家だったら、わしはおまえの家の取締役になろうものをなあ!」と。}

\sent
\re{於}{イテ}\okuri{是}{ニ}\ni{使}{ム}子\okuri{貢}{ヲシテ}\ichire{至}{ラ}\okuri{楚}{ニ}(。)
\okuri{楚}{ノ}昭王\re{興}{シテ}\okuri{師}{ヲ}\ni{迎}{ヘ}孔\ichi{子}{ヲ}(、)\okuri{然}{ル}\okuri{後}{ニ}\re{得}{タリ}\okuri{免}{ルルヲ}(。)

\yaku{かくて、子貢を使者として楚へやった。楚の昭王は兵をおこして孔子を迎えたので、やっと厄窮を免れることができた。}

\begin{flushright}{(史記 世家 孔子世家\cite{shiki})}\end{flushright}

\section{孟母三遷・孟母断機}
\sent
\okuri{鄒}{ノ}\furi{まう}{孟}\furi{か}{軻}之母也。
\ni{号}{ス}孟\ichi{母}{ト}(。)
\okuri{其}{ノ}舎\re{近}{シ}\okuri{墓}{ニ}(。)
孟子之\okuri[わか]{少}{キトキ}也、嬉\okuri{遊}{スルニ}\ni{為}{シ}墓間之\ichi{事}{ヲ}(、)\footnotekanbun{踊躍築埋}{踊躍は『礼記』問喪が規定する哭踊のこと。死者を悼む悲しみの情を鎮めるため号泣しながら跳躍する儀式。築埋は墳を作り棺を地中に入れる。}踊躍築\okuri{埋}{ス}。
孟母\okuri{曰}{ハク}(、)「\okuri{此}{レ}\kundoku{非}{}{ザル}{四}\okuri{吾}{ノ}\SAN{所}{}\okuri{以}{ニ}\NI{居}{}\okuri{処}{セシムル}\ichi{子}{ヲ}\okuri{也}{ト}(。)」
\okuri{乃}{チ}\okuri{去}{リテ}\ni{舎}{ス}\okuri{市}{ノ}\ichi{傍}{ラニ}(。)
\okuri{其}{ノ}嬉\okuri{戯}{スルニ}\ni{為}{ス}賈人\footnotekanbun{衒売}{実物の価値以上に褒め上げて売る。}\chuuskip\furi{げん}{衒}\furi{ばい}{売}之\ichi{事}{ヲ}(。)
孟母又\okuri{曰}{ハク}(、)「此\kundoku{非}{}{ザル}{四}\okuri{吾}{ノ}\SAN{所}{}\okuri{以}{ニ}\NI{居}{}\okuri{処}{セシムル}\ichi{子}{ヲ}\okuri{也}{ト}(。)」
\okuri{復}{タ}\okuri{徙}{リテ}\ni{舎}{ス}学宮之\ichi{傍}{ニ}(。)
\okuri{其}{ノ}嬉\okuri{遊}{スルニ}\chuuskip\footnotekanbun{乃}{なんと。}\okuri{乃}{チ}\ni{設}{ケ}俎\ichi{豆}{ヲ}(、)\footnotekanbun{揖譲}{腕を合わせて敬礼する。}\chuuskip\furi{いふ}{揖}\furi{じやう}{譲}進\okuri{退}{ス}(。)
孟母\okuri{曰}{ハク}(、)「\okuri{真}{ニ}\san{可}{シト}\okuri{以}{テ}\ni{居}{ラシム}\okuri{吾}{ガ}\ichi{子}{ヲ}矣。」
\okuri{遂}{ニ}\re{居}{ル}\okuri{之}{ニ}(。)
\ni{及}{ビテ}孟子\ichi{長}{ズルニ}(、)\ni{学}{ビ}六\ichi{芸}{ヲ}(、)\okuri{卒}{ニ}\ni{成}{セリ}大儒之\ichi{名}{ヲ}(。)
君子\okuri{謂}{フ}(、)「孟母\okuri{善}{ク}\chuuskip\footnotekanbun{以漸化}{環境の影響力をしだいに染み込ませて善導した。}\re{以}{テ}\okuri{漸}{ヲ}\okuri{化}{スト}(。)」
\footnotekanbun{詩}{『詩経』\UTF{9118}風・干旄の句。}\okuri{詩}{ニ}\okuri{云}{フ}(、)「\okuri{彼}{ノ}\okuri[しゅ]{\UTF{59DD}}{タル}\okuri{者}{ハ}子、\okuri{何}{ゾ}\okuri{以}{テ}\re{予}{ヘント}\okuri{之}{ニ}(。)」
此之\okuri{謂}{ヒ}也。

\yaku{鄒の孟軻の母のことである。彼女は孟母といわれた。その家は墓に近かった。孟子は幼いころ、お墓ごっこをして楽しんだ。哭踊(こくよう)を真似て跳びはねたり墳(つか)を築いては納棺の真似をするのである。孟母はいった、「ここは息子を置いておく所じゃないわ」と。そこで引きはらって市場のそばに住んだ。品物を褒めてふっかけて売る商人ごっこをして楽しんでいる。孟母は今度もいった、「ここは息子を置いておく所じゃないわ」と。ふたたび引っこして学校のそばに住んだ。ごっこ遊びはといえば、なんとお供えの膳や高坏(たかつき)をならべ、挨拶やら儀式の所作ごとの真似なのである。孟母はいった、「ほんとにここなら、わが子を置いておけるわ」と。孟子は年ごろになると、六経を学び、ついに大儒の名をなしたのであった。
君子はいう、「孟母は環境の影響力で子をしだいに善導したのである」と。『詩経』には、「あの子はなんと素直な人よ、何を上げたらよいかしら、〔躾によい場所を上げましょう〕」といっている。これは孟子が子によい環境をあたえた気持を詠っているのである。}

\sent
孟子之\okuri[わか]{少}{キトキ}也、\footnotekanbun{既学而帰}{勉強に区切りをつけて帰る。}\okuri{既}{ニ}\okuri{学}{ビテ}而\okuri{帰}{ル}(。)
孟母\okuri[まさ]{方}{ニ}\okuri{織}{レリ}(。)
\okuri{問}{ヒテ}\okuri{曰}{ハク}(、)「学\okuri{何}{レニ}\re{所}{ゾト}\okuri{至}{ル}矣。」
孟子\okuri{曰}{ハク}(、)「\footnotekanbun{自若}{もとのままである。}自\okuri{若}{タリト}也。」
孟母\re{以}{テ}\okuri[たう]{刀}{ヲ}\ni{断}{ツ}\okuri{其}{ノ}\ichi{織}{ヲ}(。)
孟子\okuri{懼}{レテ}而\ni{問}{フ}\okuri{其}{ノ}\ichi{故}{ヲ}(。)
孟母\okuri{曰}{ハク}(、)「子之\re{廃}{スルハ}\okuri{学}{ヲ}(、)\san{若}{シ}\okuri{吾}{ノ}\ni{断}{ツガ}\okuri{斯}{ノ}\ichi{織}{ヲ}也。
\okuri{夫}{レ}君\okuri{子}{ハ}\okuri{学}{ビテ}\okuri{以}{テ}\re{立}{テ}\okuri{名}{ヲ}(、)\okuri{問}{ヒテ}\okuri{則}{チ}\re{広}{ム}\okuri{知}{ヲ}(。)
\okuri{是}{ヲ}\okuri{以}{テ}\chuuskip\footnotekanbun{居}{日常生活を送る。}\okuri{居}{レバ}\okuri{則}{チ}安寧、\footnotekanbun{動}{なにか事があって行動する。}\okuri{動}{ケバ}\okuri{則}{チ}\re{遠}{ザカル}\okuri{害}{ニ}(。)
\okuri{今}{ニシテ}而\re{廃}{スルハ}\okuri{之}{ヲ}(、)是\re{不}{シテ}\ni{免}{レ}於\footnotekanbun{廝}{身分の低い人。}\chuuskip\furi{し}{廝}\ichi[えき]{役}{ヲ}、而\san{無}{キ}\okuri{以}{テ}\ni{離}{ルル}於禍\ichi{患}{ヨリ}\furi{なり}{也}(。)
\okuri{何}{ゾ}\okuri{以}{テ}\ni{異}{ナラン}於織\okuri{績}{シテ}而\okuri{食}{スルニ}、中\okuri{道}{ニシテ}\okuri{廃}{シテ}而\ichire{不}{ルニ}\okuri{為}{サ}(。)
\okuri[いづく]{寧}{ンゾ}\okuri{能}{ク}\ni[き]{衣}{セテ}\okuri{其}{ノ}夫・\ichi{子}{ニ}(、)而\okuri{長}{ク}\re{不}{ラシメン}\ni{乏}{シカラ}糧\ichi{食}{ニ}\furi{や}{哉}(。)
女\okuri{則}{チ}\ni{廃}{シ}\okuri{其}{ノ}\ichire{所}{ヲ}\okuri{食}{スル}(、)男\okuri{則}{チ}\ni[おこた]{堕}{レバ}於\ichire{脩}{ムルヲ}\okuri{徳}{ヲ}(、)\re{不}{ンバ}\ni{為}{サ}窃\ichi{盗}{ヲ}(、)\okuri{則}{チ}\ni{為}{ラント}虜\ichi{役}{ト}矣。」
孟子\okuri{懼}{レ}(、)旦夕\re{勤}{メテ}\okuri{学}{ニ}\re{不}{}\okuri{息}{マ}(。)
\NI{師}{}\okuri{事}{シテ}子\ichi{思}{ニ}(、)\okuri{遂}{ニ}\ni{成}{レリ}天下之名\ichi{儒}{ト}(。)
君子\okuri{謂}{フ}(、)「孟\okuri{母}{ハ}\shita{知}{レリト}\ni{為}{ル}\okuri{人}{ノ}\ichi{母}{}之\ue{道}{ヲ}矣。」
\footnotekanbun{詩}{『詩経』\UTF{9118}風・干旄の句。}\okuri{詩}{ニ}\okuri{云}{フ}(、)「\okuri{彼}{ノ}\okuri{\UTF{59DD}}{タル}\okuri{者}{ハ}子、\okuri{何}{ゾ}\okuri{以}{テ}\re{告}{ゲント}\okuri{之}{ニ}(。)」
此之\okuri{謂}{ヒ}也。

\yaku{孟子は幼いころ、勉強に区切りをつけて家に帰ったことがある。孟母は今しも機(はた)を織っているところであった。「どこまで上達したの」とたずねる。孟子が「あいかわらずです」と答える。孟母は剪刀(はさみ)を執ると織りかけの布を切りさいた。孟子が怕(こわ)くなって訳をたずねる。すると孟母は、「あなたが勉強をやめてしまったのは、私がこの織り物を切りさくようなものです。いったい、君子は勉強して名声を揚げ、疑点をたずねては知識を広めるものなのです。こうして普段の暮らしを安らかに、何か行動するときには禍害(わざわい)に遠ざかるようにするものなのです。いまにして勉強をやめてしまうのでは、人に使われる卑しい身分に落ちこまざるを得ないし、禍患(わざわい)から身をよけられません。これではどうして、織り績(つむ)いで生計(くらし)を立てているのに、途中でやめて仕事をしないのとちがいがありましょうか。これではどうして、夫や子に着物を着せ、いつもひもじい思いをさせずにしておくことができましょうか。女が口漱ぎ(くちすぎ)の仕事をやめ、男が徳の修養を怠るならば、窃盗(ぬすみ)をしなければ、人の卑しい使用人となるしかありません」といって聴かせた。孟子は怕くなって、毎朝毎晩、やすむことなく勉強にこれつとめたのであった。子思(実はその門人)を師として学び、ついに天下の名儒となったのである。
君子はいう、「孟母は人の母たるものの道を心得ていた」と。『詩経』には、「あの子はなんと素直な人よ、何を告げたらよいかしら、〔譬(たと)えで説教しときましょう〕」といっている。これは孟母が子をしゅんとさせ、断機の譬えをもって修業の意味を教えたことを詠っているのである。}

\sent
孟子\okuri{既}{ニ}\okuri{娶}{リ}(、)\re{将}{ニ}\ni{入}{ラント}\chuuskip\footnotekanbun{私室}{寝室。}私\ichi{室}{ニ}(、)\okuri{其}{ノ}婦\footnotekanbun{袒}{肌ぬぎになる。}\chuuskip\okuri[かたぬ]{袒}{ギシテ}而\re{在}{リ}\okuri{内}{ニ}(、)孟子\re{不}{}\okuri{悦}{バ}(、)\okuri{遂}{ニ}\okuri{去}{リテ}\re{不}{}\okuri{入}{ラ}(。)
婦\ni{辞}{シテ}孟\ichi{母}{ニ}而\re{求}{メテ}\okuri{去}{ランコトヲ}(、)\okuri{曰}{ハク}(、)
「妾\okuri{聞}{ク}(、)『夫婦之\okuri{道}{ハ}(、)私室\re{不}{ト}\okuri{与}{ラ}焉。』
今者、妾\okuri{窃}{カニ}\okuri[おこた]{堕}{リテ}\re{在}{ルニ}\okuri{室}{ニ}(、)\okuri[すなは]{而}{チ}夫子\re{見}{テ}\okuri{妾}{ヲ}(、)\footnotekanbun{勃然}{むっとする。}勃\okuri{然}{トシテ}\re{不}{}\okuri{悦}{バ}(、)是\footnotekanbun{客}{他所者扱いをする。}\re{客}{トスル}\okuri{妾}{ヲ}也。
婦人之義、\okuri{蓋}{シ}\ni{不}{}\footnotekanbun{客宿}{他所に泊まる。}客\ichi{宿}{セ}(。)
\okuri{請}{フラクハ}\ni{帰}{セント}父\ichi{母}{ニ}(。)」
\re{於}{イテ}\okuri{是}{ニ}(、)孟母\ni[よ]{召}{ビテ}孟\ichi{子}{ヲ}(、)而\re{謂}{ヒテ}\okuri{之}{ニ}\okuri{曰}{ハク}(、)
「\okuri{夫}{レ}\okuri{礼}{ニ}\re{将}{ニ}<ルニ>\re{入}{ラント}\okuri{門}{ニ}(、)\ni{問}{フハ}\okuri[たれ]{孰}{カ}\ichi{存}{スト}(、)\NI{所}{}以\ichire{致}{ス}\okuri{敬}{ヲ}也。
\re{将}{ニ}<ルニ>\re{上}{ラント}\okuri{堂}{ニ}(、)声\okuri{必}{ズ}\okuri{揚}{グルハ}(、)\NI{所}{}以\footnotekanbun{戒}{警戒させる。}\ichire{戒}{ムル}\okuri{人}{ヲ}也。
\re{将}{ニ}<ルニ>\re{入}{ラント}\okuri{戸}{ニ}(、)\footnotekanbun{視}{視線。}\okuri{視}{ヲバ}\okuri{必}{ズ}\okuri{下}{グルハ}(、)\re{恐}{ルル}\ni{見}{ンコトヲ}\okuri{人}{ノ}\ichi{過}{チヲ}也。
今\okuri{子}{ハ}\re{不}{シテ}\ni{察}{セ}於\ichi{礼}{ヲ}(、)而\ni{責}{ム}\okuri{礼}{ヲ}於\ichi{人}{ニ}(、)\footnotekanbun{不亦遠乎}{なんと道理にはずれていることであろう。}\ni{不}{}亦\ichi{遠}{カラ}\okuri[や]{乎}{ト}(。)」
孟子\okuri{謝}{シテ}(、)\okuri{遂}{ニ}\ni{留}{ム}\okuri{其}{ノ}\ichi{婦}{ヲ}(。)
君子\okuri{謂}{フ}(、)「孟母\re{知}{リテ}\okuri{礼}{ヲ}(、)而\ni{明}{ラカナリト}於姑母之\ichi{道}{ニ}(。)」

\yaku{孟子が結婚後、私室に入ろうとしたときのことである。婦(よめ)がしどけなく肌ぬぎになって中にいた。孟子は不機嫌になって、その場を去って入らなかった。婦(よめ)は孟母のもとに暇(いとま)乞いにゆき、離縁を求めて、「『夫婦の礼は、私室にては関係なし』と聞いておりました。ところが今、妾(わたし)が一人で部屋でくつろいでおりましたら、旦那さまが妾(わたし)をご覧になり、むっとして不機嫌になられました。これでは妾(わたし)は他所者(よそもの)あつかいです。婦人の礼は、おもうに他所(よそ)に泊まったりはしません。父母のもとに帰してください」という。そこで孟母は孟子をよんで彼にむかって、「いったい礼では、門口に入ろうとするときは、誰かおいでかとたずねるのは、相手に敬意をはらうためなのです。堂(へや)にのぼろうとするときは、必ず声をかけるのは相手に注意を促すためなのです。戸口に入ろうとするとき、目を必ず伏せるのは、人の過ちを見るのを恐れるためなのです。なのに今あなたは、礼をよくわきまえもせず、礼を人に求めて責めています。なんと道理にはずれていることでしょう。」といって聴かせた。孟子は詫びをいれて、婦(よめ)を家にとどめた。
君子はいう、「孟母は礼を心得、姑母(しゅうとめ)の道に明るかった」と。}

\sent
孟子\re{処}{レドモ}\okuri{斉}{ニ}(、)而\ni{有}{リ}憂\ichi{色}{}(。)
孟母\re{見}{テ}\okuri{之}{ヲ}\okuri{曰}{ハク}(、)「子\re{若}{シ}\ni{有}{ルガ}憂\ichi{色}{}(、)\okuri{何}{ゾ}\okuri{也}{ト}(。)」
孟子\okuri{曰}{ハク}(、)「\okuri[しからざ]{不}{ル}\okuri{也}{ト}(。)」
異日\footnotekanbun{間居}{暇なとき、また、人を避けて一人でいるとき。}間\okuri{居}{セシトキ}(、)\re{擁}{シテ}\chuuskip\footnotekanbun{楹}{丸く太い柱。}\chuuskip\okuri[はしら]{楹}{ヲ}而\okuri{歎}{ズ}(。)
孟母\re{見}{テ}\okuri{之}{ヲ}\okuri{曰}{ハク}(、)「\okuri[さき]{郷}{ニ}\san{見}{ルニ}\okuri{子}{ノ}\ni{有}{ルヲ}憂\ichi{色}{}(、)\ni{曰}{ヘリ}『\okuri[しからざ]{不}{ル}\okuri{也}{ト}(。)』今\re{擁}{シテ}\okuri{楹}{ヲ}而\okuri{歎}{ズルハ}(、)\okuri{何}{ゾ}\okuri{也}{ト}(。)」
孟子\okuri{対}{ヘテ}\okuri{曰}{ハク}(、)「軻\re{聞}{ケリ}\okuri{之}{ヲ}(。)『君\okuri{子}{ハ}\re[かな]{称}{ヒテ}\okuri{身}{ニ}而\re{就}{ク}\okuri{位}{ニ}(。)\re{不}{}\ni{為}{サ}\okuri[かりそ]{苟}{メニ}\ichi{得}{ルヲ}(。)\okuri{而}{モ}\re{受}{クルモ}\okuri{賞}{ヲ}(、)\re{不}{}\ni{貪}{ラ}栄\ichi{禄}{ヲ}(。)
諸侯\re{不}{ンバ}\okuri{聴}{カ}(、)\okuri{則}{チ}\re{不}{}\footnotekanbun{達}{意見を上進する。}\ni{達}{セ}\okuri{其}{ノ}\ichi{上}{ニ}(。)
\okuri{聴}{ケドモ}而\re{不}{ンバ}\okuri{用}{ヒ}(、)\okuri{則}{チ}\re{不}{ト}\ni{践}{マ}\okuri{其}{ノ}\ichi{朝}{ヲ}(。)』
今\okuri{道}{ハ}\re{不}{}\ni{用}{ヒラレ}於\ichi{斉}{ニ}(。)\re{願}{ヘドモ}\okuri[さ]{行}{ラント}而\okuri{母}{ハ}\okuri{老}{ユ}(。)\okuri{是}{ヲ}\okuri{以}{テ}\okuri{憂}{フル}\okuri{也}{ト}(。)」
孟母\okuri{曰}{ハク}(、)「\okuri{夫}{レ}婦人之\okuri{礼}{ハ}(、)\ni{精}{シ}五\ichi{飯}{ヲ}(、)\footnotekanbun{幎酒漿}{酒や飲み物を布巾で覆って整える。}\chuuskip\ni[べき]{幎}{シ}酒\ichi{漿}{ヲ}(、)\ni{養}{ヒ}舅\ichi{姑}{ヲ}(、)\ni{縫}{フ}衣\ichi{裳}{ヲ}而\okuri{已}{ナリ}矣。
\okuri{故}{ニ}\ni{有}{リテ}\chuuskip\footnotekanbun{閨内}{女性の家事生活空間。奥向き。}閨内之\ichi{脩}{}(、)而\ni{無}{シ}\footnotekanbun{境外}{家の境の外。}境外之\ichi{志}{}(。)
\footnotekanbun{易}{『易経』家人・六二 爻辞。}\okuri{易}{ニ}\okuri{曰}{フ}(、)『\ni{在}{リ}\footnotekanbun{中饋}{家の中の飲食物。}\chuuskip\furi{ちゅう}{中}\ichi[き]{饋}{ニ}(、)\re{無}{シト}\re[ところ]{攸}{}\okuri{遂}{グル}(。)』
\footnotekanbun{詩}{『詩経』小雅・斯干。}\okuri{詩}{ニ}\okuri{曰}{フ}(、)『\re{無}{ク}\footnotekanbun{非}{悪事。}\okuri{非}{ト}\re{無}{ク}\footnotekanbun{儀}{善事。}\okuri{儀}{ト}(、)\okuri{惟}{ダ}酒\okuri{食}{ノミ}是\okuri{議}{ルト}(。)』
\okuri{以}{テ}\shita{言}{フ}婦人\ni{無}{クシテ}\chuuskip\footnotekanbun{擅制}{独断で思うがままに物事を処置する。}\chuuskip\furi{せん}{擅}\furi{せい}{制}之\ichi{義}{}(、)而\naka{有}{ルヲ}三従之\ue{道}{}也。
\okuri{故}{ニ}年\okuri{少}{ケレバ}\okuri{則}{チ}\ni{従}{ヒ}乎父\ichi{母}{ニ}(、)\okuri{出}{デ}\okuri{嫁}{シテハ}\okuri{則}{チ}\ni{従}{ヒ}乎\ichi{夫}{ニ}(、)夫\okuri{死}{シテハ}\okuri{則}{チ}\ni{従}{フハ}乎\ichi{子}{ニ}(、)礼也。
今\okuri{子}{ハ}成人也、\okuri{而}{ルニ}我\okuri{老}{イタリ}矣。
\okuri{子}{ハ}\ni{行}{ヘ}乎\okuri{子}{ノ}\ichi{義}{ヲ}(。)\okuri{吾}{ハ}\ni{行}{ハント}乎\okuri{吾}{ノ}\ichi{礼}{ヲ}(。)」
君子\okuri{謂}{フ}(、)「孟\okuri{母}{ハ}\ni{知}{レリト}婦\ichi{道}{ヲ}(。)」
\footnotekanbun{詩}{『詩経』魯頌・\UTF{6CEE}水。}\okuri{詩}{ニ}\okuri{云}{フ}(、)「\okuri[すなは]{載}{チ}\okuri{色}{シ}\okuri{載}{チ}\okuri{笑}{フ}(。)\re{匪}{ズ}\okuri{怒}{ルニ}\re{匪}{ズト}\okuri{教}{フルニ}(。)」
此之\okuri{謂}{ヒ}也。

\yaku{孟子は斉にいたが浮かぬ顔をしていた。孟母がこれを見て、「あなたは浮かぬ顔のようね。なぜなの」という。孟子は、「いいえ」といった。他日、〔孟子が〕何することなく楹(はしら)にすがって歎息をついていたときのことだ。孟母はこれを見て、「まえに、あなたが浮かぬ顔をしていたのを見たとき、『いいえ』といったわね。なのに今楹(はしら)にすがって歎息をついているのは、なぜなの」という。孟子は、「『君子は身のほど相応に位に就く。いいかげんには禄を手にせぬものだ。賞を受けても高禄をむさぼったりはせぬ。諸侯が自説を聴いてくれねば、意見を上奏しない。聴いてくれても採用してくれぬなら、その朝廷には出かけたりはしない』と聞いています。でも今は、わが道は斉に採用されません。他所国(よそぐに)へ行きたいと願っても、お母さまがお年を召しておいでです。そこで浮かぬ顔をしておりました」とこたえた。孟母はそこで、「いったい婦人の礼は、五穀の飯の穀粒をとぎ、酒や飲みものをととのえ、舅(しゅうと)・姑(しゅうとめ)のお世話をし、上着や裳(はかま)を縫ったりすることに他なりません。だから奥向きの仕事の修業はあっても、家庭外の仕事への意志はありません。『易経』には、「女の任務(つとめ)は家事・食事。おのれの意向(おもい)を遂げぬこと」といっていますし、『詩経』には、「悪事も善事もなしはせず、酒の仕度(したく)に料理に励め」といっています。そこで婦人には自分で専断する義はなく、三従の道があるともいわれるのです。だから子どもの頃は父母に従い、嫁いだならば夫に従い、夫が死ねば子に従うのが礼なのです。今やあなたは成人(おとな)なのです。しかも私は年をとりました。あなたはあなたの義を行いなさい。私は私の礼を行いましょう」といって聴かせたのであった。
君子はいう、「孟母は婦道を心得ていた」と。『詩経』には、「顔色やわらげにっこり笑う。ことさら怒らず教えもしない。〔おのずと導くその偉さ〕」という。これはおだやかに道理を説いて孟子を悟らせ導いた孟母のことを詠っているのである。}

\sent
\okuri{頌}{ニ}\okuri{曰}{ハク}(、)「孟子之母、教化\footnotekanbun{列分}{葬儀屋や坐商とは異なる士大夫の身分の列の中に子を入れるように教化した。}\re{列}{ヌ}\okuri{分}{ヲ}(、)\re{処}{ラシムルニ}\okuri{子}{ヲ}\re{択}{ビ}\okuri{芸}{ヲ}(、)\re{使}{ム}\ni{従}{ハ}大\ichi{倫}{ニ}(。)\okuri{子}{ノ}学\re{不}{レバ}\okuri{進}{マ}(、)\re{断}{チテ}\okuri{機}{ヲ}\re{示}{ス}\okuri{焉}{ヲ}(。)\okuri{子}{ハ}\okuri{遂}{ニ}\re{成}{シテ}\okuri{徳}{ヲ}(、)\ni{為}{レリト}当\okuri{世}{ノ}\chuuskip\footnotekanbun{冠}{第一等の儒者。}\ichi{冠}{ト}(。」)

\yaku{頌にいう、「孟子の母なる賢(すぐ)れし女(ひと)は、教えて分(ぶん)のけじめつけたり。子の住居に習うべき芸(わざ)えらびて、大道に就き従わせたり。子の学の進まぬときは、機織る(はたおる)布断ち誤り示せり。子はついに徳器を育てて、当世一の儒者とはなれり」と。}

\begin{flushright}{(列女伝 鄒孟軻母\cite{retsujoden})}\end{flushright}

\vfill\setstretch{1}
\begin{thebibliography}{9}
	\bibitem{kanbuntaikei}林秀一.新釈漢文大系 第\rensuji{20}巻 十八史略(上).第\rensuji{2}版,明治書院,一九六七.
	\bibitem{shoukou7}著 杜預,註 蔡済恭,李秉模,編校 李書九.春秋左氏伝 第\rensuji{19}巻.\url{http://www.wul.waseda.ac.jp/kotenseki/html/ro12/ro12_01772/index.html},二〇一九年二月二十七日参照.
	\bibitem{shiki}撰 司馬遷,集解 裴\UTF{99F0}.史記.\url{http://www.wul.waseda.ac.jp/kotenseki/html/ri08/ri08_01735_0001/index.html},二〇一九年二月二十六日参照.
	\bibitem{kanbuntaikei_shiki}吉田賢抗.新釈漢文大系 第\rensuji{87}巻 史記 七 (世家下).初版,明治書院,一九八二.
	\bibitem{retsujoden} 撰 劉向.列女伝.\url{https://zh.wikisource.org/wiki/列女傳},二〇一九年二月二十六日参照.
	\bibitem{retsujoden_meiji}山崎純一.新編漢文選 思想・歴史シリーズ 列女伝 上.明治書院,一九九六.ISBN 978-4-625-57204-3.
	\bibitem{retsujoden_heibon}中島みどり.東洋文庫 列女伝 \rensuji{1}.初版,平凡社,二〇〇一.ISBN 978-4-582-80686-4.
\end{thebibliography}

\end{document}