\documentclass[a4j,twocolumn]{tarticle}

\usepackage{sfkanbun}
\usepackage{otf}
\usepackage[dvipdfmx]{graphicx} %図の挿入
\usepackage{url}
\usepackage{setspace} %行間の部分的変更

\setlength{\columnseprule}{0.1pt}

\def\ninojiten{%
 \kern-1zw%全角空白
\raise.2zw\hbox to1zw{\smash{\kern-.1zw\hbox to0zw{%
{\scriptsize ゝ\kern-.5zw\smash{\raise.1zh\hbox{ヽ}}}\hss}}\hss}%
\kern-1zw  %全角空白
}
\def\:{\rotatebox[origin=c]{-90}{:}}
\def\;{\rotatebox[origin=c]{-90}{;}}

\newcommand{\ten}[2]{\kundoku{#1}{}{}{}(#2)}
\newcommand{\okuri}[3][]{\kundoku{#2}{#1}{#3}{}}
\newcommand{\furi}[2]{\kundoku{#1}{#2}{}{}}
\newcommand{\re}[3][]{\kundoku{#2}{#1}{#3}{レ}}
\newcommand{\ichire}[3][]{\kundoku{#2}{#1}{#3}{\ichireten}}
\newcommand{\ichi}[3][]{\kundoku{#2}{#1}{#3}{一}}
\renewcommand{\ni}[3][]{\kundoku{#2}{#1}{#3}{二}}
\newcommand{\san}[3][]{\kundoku{#2}{#1}{#3}{三}}
\newcommand{\ue}[3][]{\kundoku{#2}{#1}{#3}{上}}
\newcommand{\shita}[3][]{\kundoku{#2}{#1}{#3}{下}}
\newcommand{\uere}[3][]{\kundoku{#2}{#1}{#3}{\uereten}}
\newcommand{\NI}[3][]{\kundoku{#2}{#1}{#3}{\ongofu{二}{---}}}
\newcommand{\SAN}[3][]{\kundoku{#2}{#1}{#3}{\ongofu{三}{--}}}

%\newcommand{\LRkanji}[2]{\scalebox{0.5}[1]{\makebox[2zw]{#1\hspace{-0.1zw}#2}}}
\newcommand{\chuu}[1]{\begin{quote} \footnotesize{#1} \end{quote}}
\newcommand{\yaku}[1]{\begin{quote} \small{\textgt{訳}#1} \end{quote}}
\newcommand{\nw}[1]{{ \bf #1 }}
\newcommand{\kin}{\raisebox{-0.5zw}{\rotatebox{90}{\includegraphics[width=1zw]{Kin_2.pdf}}}} %出力できない文字の埋め込み
\newcommand{\kingt}{\raisebox{-0.5zw}{\rotatebox{90}{\includegraphics[width=1zw]{Kin_1.pdf}}}} %出力できない文字の埋め込み

\newcounter{kanumber}
\setcounter{kanumber}{0}

%\newcounter{sentencenumber}
%\setcounter{sentencenumber}{0}
%\newcommand{\numbun}{\refstepcounter{sentencenumber}\noindent\textsf{\rensuji{\thesentencenumber}}\hspace{1zw}}

%入力参考
%\kundoku{字}{よみ}{送り}{訓点}<左送り>(句読点)

\newenvironment{xia}[1]{\refstepcounter{kanumber}第\rensuji{\thekanumber}代\nw{#1} \begin{itemize}\vspace{-0.5\baselineskip}}{\vspace{-0.5\baselineskip}\end{itemize}}

\begin{document}
\begin{center}{\LARGE {\bf 十八史略ゼミ 発表補足資料}}\\夏\end{center}
\vspace{-\baselineskip}
\begin{flushright}
第一版 \rensuji{2018}年\rensuji{12}月\rensuji{23}日\\
文責\:ト部蛸焼
\end{flushright}
\vspace{-1.5\baselineskip}

\section{本文の概略}
\begin{xia}{禹}
\item 鯀の子、\UTF{9853}\UTF{980A}の孫。
\item 鯀の治水事業を後継し成功。その功績により舜を後継。
\item 一饋十起。
\item 罪人を見て自身の徳を顧みる。
\item 酒を作った儀狄を忌避する。
\item 鼎を作って鬼神をもてなす。
\item 黄竜にも毅然と振る舞う。
\end{xia}

\begin{xia}{啓}
\item 禹の子。(初の世襲\; 益との後継者問題)
\item 甘誓 (\rensuji{cf.}本稿\bf{\ref{甘誓}})
\end{xia}

\begin{xia}{太康}
\item 啓の子。
\item 放蕩のせいで后\UTF{7FBF}にやめさせられる。
\end{xia}

\begin{xia}{仲康}
\item 太康の弟。
\item 后\UTF{7FBF}の傀儡政治。羲和の征伐。
\end{xia}

\begin{xia}{相}
\item 仲康の子。
\item 追放されて后\UTF{7FBF}の政権$\to$寒\UTF{6D5E}の奪権
\end{xia}

\begin{xia}{少康}
\item 相の子。
\item 旧臣の靡とともに寒\UTF{6D5E}を追放。王家の復権。
\end{xia}

\begin{xia}{杼}
\item 少康の子。
\end{xia}

\begin{xia}{槐}
\item 杼の子。
\end{xia}

\begin{xia}{芒}
\item 槐の子。
\end{xia}

\begin{xia}{泄}
\item 芒の子。
\end{xia}

\begin{xia}{不降}
\item 泄の子。
\end{xia}

\begin{xia}{\UTF{6243}}
\item 不降の弟。
\end{xia}

\begin{xia}{\kingt}
\item \UTF{6243}の子。
\end{xia}

\begin{xia}{孔甲}
\item \kin の子。
\item 鬼神を好む。淫乱を極める。
\item 劉累特製の竜の塩漬けをうましうましと食べて逃げられる。
\end{xia}

\begin{xia}{皐}
\item 孔甲の子。
\end{xia}

\begin{xia}{発}
\item 皐の子。
\end{xia}

\begin{xia}{履癸}
\item 発の子。桀。貪虐。
\item 末喜のための傾宮瑶台。
\item 肉山脯林の宴会。
\item 鳴条の戦い。(\rensuji{cf.}本稿\bf{\ref{鳴条}})
\end{xia}

\section{甘誓\protect\footnote{甘誓は『書経』に収録されている。\cite{kansei}}}\label{甘誓}
\okuri{大}{イニ}\ni{戦}{ヒ}于\ichi{甘}{ニ}(、)\okuri{乃}{チ}\ni{召}{ス}六\ichi{卿}{ヲ}(。)
王\okuri{曰}{ハク}(、)
「嗟、六事之人、予\NI{誓}{} \okuri{告}{ス}\ichi{汝}{ニ}(。)
有扈氏\NI{威}{}\okuri{侮}{シ}五\ichi{行}{ヲ}(、)\NI{怠}{}\okuri{棄}{シ}三\ichi{正}{ヲ}(、)
天\okuri[もつ]{用}{テ}\NI{勦}{}\okuri{絶}{ス}\okuri{其}{ノ}\ichi{命}{ヲ}(。)
今予\okuri{惟}{レ}\NI{恭}{}\okuri{行}{ス}天之\ichi{罰}{ヲ}(。)
左\re{不}{レバ}\ni[おさ]{攻}{メ}于\ichi{左}{ニ}(、)汝\re{不}{}\re{恭}{セ}\okuri{命}{ヲ}(。)
右\re{不}{レバ}\ni[おさ]{攻}{メ}于\ichi{右}{ニ}(、)汝\re{不}{}\re{恭}{セ}\okuri{命}{ヲ}(。)
御\ni{非}{ザレバ}\okuri{其}{ノ}馬之\ichi{正}{シキニ}(、)汝\re{不}{}\re{恭}{セ}\okuri{命}{ヲ}(。)
\re{用}{フルハ}\okuri{命}{ヲ}(、)\ni{賞}{シ}于\ichi{祖}{ニ}(、)
\re{弗}{ルハ}\re{用}{ヒ}\okuri{命}{ヲ}(、)\ni{戮}{ス}于\ichi{社}{ニ}(。)
予\okuri{則}{チ}\NI{孥}{}\okuri{戮}{セント}\ichi{汝}{ヲ}(。)」

\yaku{\cite{kansei-yaku} 夏の王、啓が、甘で大戦闘を開始しようとするに先だち、
六卿たちを呼びよせて告げた。

王はいった、

「ああ、六事の人々よ。

予は、神々に誓って、汝たちに告げるぞ。有扈氏がおぞましくも五行をないがしろにして、
三正を破り棄てたので、天が有扈氏の国運を断絶したのだ。されば、今、予は天の刑罰を奉行しようとしているのだ。

戦闘にあたり、汝たちの車左がその任務の射を怠るならば、それは汝たちが天の命令に従っていないのだ。
車右がその任務の戈をふるうことを怠るならば、それは汝たちが天の命令に従っていないのだ。
御者が馬の進退を誤るならば、それは汝たちが天の命令に従っていないのだ。

この予が命令に従った、これを祖廟で賞するであろう。
命令に従わなかった者は、これを社で罰するであろう。
予はさらに汝たちを奴隷におとして辱めるであろう。」と。}


\section{鳴条の戦い\protect\footnote{この文章は『史記』の「夏本紀」から抜粋した。\cite{meijou}}} \label{鳴条}


孔甲\okuri{崩}{レ}(、)\okuri{子}{ノ}帝皐\okuri{立}{ツ}(。)
帝皐\okuri{崩}{レ}(、)\okuri{子}{ノ}帝発\okuri{立}{ツ}(。)
帝発\okuri{崩}{レ}(、)\okuri{子}{ノ}帝履癸\okuri{立}{チ}(、)
\okuri{是}{ヲ}\re{為}{ス}\okuri{桀}{ト}(。)
帝桀之時、
\ni{自}{リシテ}孔甲以\ichi{来}{}而
諸侯\re{多}{ク}\re[そむ]{畔}{クコト}\okuri{夏}{ニ}(、)
桀\re{不}{シテ}\re{務}{メ}\okuri{徳}{ニ}而\NI{武}{}\okuri{傷}{シ}百\ichi{姓}{ヲ}(、)
百姓\re{弗}{}\okuri{堪}{エ}(。)
\okuri{乃}{チ}\re{召}{スモ}\okuri{湯}{ヲ}而\ni{囚}{ヘ}\okuri{之}{ヲ}夏\ichi{台}{ニ}(、)
\okuri{已}{ニシテ}而\re{釈}{ス}\okuri{之}{ヲ}(。)
湯\re{修}{メ}\okuri{徳}{ヲ}(、)諸侯皆\re{帰}{シ}\okuri{湯}{ニ}(、)
湯\okuri{遂}{ニ}\re{率}{イテ}\okuri{兵}{ヲ}\okuri{以}{テ}\ni{伐}{ツ}夏\ichi{桀}{ヲ}(。)
桀\ni[に]{走}{ゲ}鳴\ichi{条}{ニ}(、)\okuri{遂}{ニ}\okuri{放}{チテ}而\okuri{死}{ス}(。)
桀\re{謂}{ヒテ}\okuri{人}{ニ}\okuri{曰}{ハク}(、)
「吾\shita{悔}{ユ}\san{不}{}\okuri{遂}{ニ}\ni{殺}{サ}\okuri{湯}{ヲ}於夏\ichi{台}{ニ}(、)
\uere{使}{ムルコトヲ}\re{至}{ラ}\okuri{此}{ニ}(。)」
湯\okuri{乃}{チ}\ni{践}{ミ}天\okuri{子}{ノ}\ichi{位}{ヲ}(、)
\ni{代}{ハリテ}夏\ichi{朝}{ニ}天\okuri{下}{ス}(。)
湯\ni{封}{ジ}夏之\ichi{後}{ヲ}(、)\re{至}{ルマデ}\okuri{周}{ニ}\ni{封}{ズ}於\ichi{杞}{ニ}也。

\yaku{\footnote{ト部訳} 孔甲は崩御し、(孔甲の)子の帝皐が帝位に就いた。
帝皐は崩御し、(帝皐の)子の帝発が帝位に就いた。
帝発は崩御し、(帝発の)子の帝履癸が帝位に就き、(死後)これを桀とした。
桀の時代は、孔甲の時代以来、諸侯が夏王朝に背くことは多かったが、
桀は徳に励まず多くの人々を武力で傷つけ(たので)、人々は我慢できなかった。
そこで、湯を呼び(助けを求めたが)、(桀は)彼を夏台にとじこめ、
しばらくすると彼を釈放した。
湯は徳を修め、諸侯はみな湯に帰属した。
湯はそのまま兵を率いて、夏の桀を攻めた。
桀は鳴条に敗走し、そのまま追放されて死んだ。
桀が人に言ったところでは、
「私は湯を夏台に(捕らえたときに)そのまま殺してしまわず、
このような(末路に)至らせてしまったことを後悔している。」と。
湯はそのあと天子の位に就き、夏王朝に代わって天下をとった。
湯は夏の後を封じ、周代に至るまで杞に都を置いた。}

\begin{thebibliography}{9}
\bibitem{kansei} 「尚書\:夏書\:甘誓 - 中國哲學書電子化計劃」、\url{https://ctext.org/shang-shu/speech-at-gan/zh} 、参照\:\rensuji{2018}年\rensuji{12}月\rensuji{6}日
\bibitem{kansei-yaku} 赤塚忠,「中国古典文学大系 全\rensuji{60}巻 書経・易経(抄) 第\rensuji{1}巻」,平凡社,\rensuji{1974}
\bibitem{meijou} 「史記\:書 - 中國哲學書電子化計劃」、\url{https://ctext.org/shiji/shu/zh} 、 参照\:\rensuji{2018}年\rensuji{12}月\rensuji{6}日
\end{thebibliography}

\end{document}